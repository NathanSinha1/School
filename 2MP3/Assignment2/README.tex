\documentclass[12pt]{article}
\usepackage{ragged2e} % load the package for justification
\usepackage{hyperref}
\usepackage[utf8]{inputenc}
\usepackage{pgfplots}
\usepackage{tikz}
\usetikzlibrary{fadings}
\usepackage{filecontents}
\usepackage{multirow}
\usepackage{amsmath}
\pgfplotsset{width=10cm,compat=1.17}
\setlength{\parskip}{0.75em} % Set the space between paragraphs
\usepackage{setspace}
\setstretch{1.2} % Adjust the value as per your preference
\usepackage[margin=2cm]{geometry} % Adjust the margin
\setlength{\parindent}{0pt} % Adjust the value for starting paragraph
\usetikzlibrary{arrows.meta}
\usepackage{mdframed}
\usepackage{float}

\usepackage{hyperref}

% to remove the hyperline rectangle
\hypersetup{
	colorlinks=true,
	linkcolor=black,
	urlcolor=blue
}

\usepackage{xcolor}
\usepackage{titlesec}
\usepackage{titletoc}
\usepackage{listings}
\usepackage{tcolorbox}
\usepackage{lipsum} % Example text package
\usepackage{fancyhdr} % Package for customizing headers and footers



% Define the orange color
\definecolor{myorange}{RGB}{255,65,0}
% Define a new color for "cherry" (dark red)
\definecolor{cherry}{RGB}{148,0,25}
\definecolor{codegreen}{rgb}{0,0.6,0}



%%%%%%%%%%%%%%%%%%%%%%%%%%%%%%%%%%%%%%%%%%%%%%%%%%%%%%%%%%%%%%%%%%%%%
% Apply the custom footer to all pages
\pagestyle{fancy}

% Redefine the header format
\fancyhead{}
\fancyhead[R]{\textcolor{orange!80!black}{\itshape\leftmark}}

\fancyhead[L]{\textcolor{black}{\thepage}}


% Redefine the footer format with a line before each footnote
\fancyfoot{}
\fancyfoot[C]{\footnotesize P. Pasandide, McMaster University, Programming for Mechatronics - MECHTRON 2MP3. \footnoterule}

% Redefine the footnote rule
\renewcommand{\footnoterule}{\vspace*{-3pt}\noindent\rule{0.0\columnwidth}{0.4pt}\vspace*{2.6pt}}

% Set the header rule color to orange
\renewcommand{\headrule}{\color{orange!80!black}\hrule width\headwidth height\headrulewidth \vskip-\headrulewidth}

% Set the footer rule color to orange (optional)
\renewcommand{\footrule}{\color{black}\hrule width\headwidth height\headrulewidth \vskip-\headrulewidth}

%%%%%%%%%%%%%%%%%%%%%%%%%%%%%%%%%%%%%%%%%%%%%%%%%%%%%%%%%%%%%%%%%%%%%


% Set the color for the section headings
\titleformat{\section}
{\normalfont\Large\bfseries\color{orange!80!black}}{\thesection}{1em}{}

% Set the color for the subsection headings
\titleformat{\subsection}
{\normalfont\large\bfseries\color{orange!80!black}}{\thesubsection}{1em}{}

% Set the color for the subsubsection headings
\titleformat{\subsubsection}
{\normalfont\normalsize\bfseries\color{orange!80!black}}{\thesubsubsection}{1em}{}


%%%%%%%%%%%%%%%%%%%%%%%%%%%%%%%%%%%%%%%%%%%%%%%%%%%%%%%%%%%%%%%%%%%%%
% Set the color for the table of contents
\titlecontents{section}
[1.5em]{\color{orange!80!black}}
{\contentslabel{1.5em}}
{}{\titlerule*[0.5pc]{.}\contentspage}

% Set the color for the subsections in the table of contents
\titlecontents{subsection}
[3.8em]{\color{orange!80!black}}
{\contentslabel{2.3em}}
{}{\titlerule*[0.5pc]{.}\contentspage}

% Set the color for the subsubsections in the table of contents
\titlecontents{subsubsection}
[6em]{\color{orange!80!black}}
{\contentslabel{3em}}
{}{\titlerule*[0.5pc]{.}\contentspage}


%%%%%%%%%%%%%%%%%%%%%%%%%%%%%%%%%%%%%%%%%%%%%%%%%%%%%%%%%%%%%%%%%%%%%
% set a format for the codes inside a box with C format
\lstset{
	language=C,
	basicstyle=\ttfamily,
	backgroundcolor=\color{blue!5},
	keywordstyle=\color{blue},
	commentstyle=\color{codegreen},
	stringstyle=\color{red},
	showstringspaces=false,
	breaklines=true,
	frame=single,
	rulecolor=\color{lightgray!35}, % Set the color of the frame
	numbers=none,
	numberstyle=\tiny,
	numbersep=5pt,
	tabsize=1,
	morekeywords={include},
	alsoletter={\#},
	otherkeywords={\#}
}




%\input listings.tex



% Define a command for inline code snippets with a colored and rounded box
\newtcbox{\codebox}[1][gray]{on line, boxrule=0.2pt, colback=blue!5, colframe=#1, fontupper=\color{cherry}\ttfamily, arc=2pt, boxsep=0pt, left=2pt, right=2pt, top=3pt, bottom=2pt}




\tikzset{%
	every neuron/.style={
		circle,
		draw,
		minimum size=1cm
	},
	neuron missing/.style={
		draw=none, 
		scale=4,
		text height=0.333cm,
		execute at begin node=\color{black}$\vdots$
	},
}



%%%%%%%%%%%%%%%%%%%%%%%%%%%%%%%%%%%%%%%%%%%%%%%%%%%%%%%%%%%%%%%%%%%%%

% Define a new tcolorbox style with default options
\tcbset{
	myboxstyle/.style={
		colback=orange!10,
		colframe=orange!80!black,
	}
}

% Define a new tcolorbox style with default options to print the output with terminal style


\tcbset{
	myboxstyleTerminal/.style={
		colback=blue!5,
		frame empty, % Set frame to empty to remove the fram
	}
}

\mdfdefinestyle{myboxstyleTerminal1}{
	backgroundcolor=blue!5,
	hidealllines=true, % Remove all lines (frame)
	leftline=false,     % Add a left line
}


\begin{document}
	
	\justifying
	
	\begin{center}
		\textbf{{\large ASSIGNMENT 2}}
		
		\textbf{Developing a Basic Genetic Optimization Algorithm in C} 
		
		NATHAN SINHA
	\end{center}	
	
	\section{MAKEFILE}
    \begin{lstlisting}
1     CC = gcc
2     CFLAGS = -Wall -Wextra -std=c99
3     LIBS = -lm
4     SOURCES = GA.c OF.c functions.c
5     OBJECTS = $(SOURCES:.c=.o)
6     EXECUTION = GA
7
8     all: $(EXECUTION)
9 
10    $(EXECUTION): $(OBJECTS)
11      $(CC) $(OBJECTS) -o $(EXECUTION) $(LIBS)
12
13    %.o: %.c
14	     $(CC) $(CFLAGS) -c $< -o $@
15
16    clean:
17	     rm -f $(OBJECTS) $(EXECUTION)
	\end{lstlisting}
 Line 1 sets the compiler to 'gcc'.\newline
 Line 2 picks the compiler flags. This will mark errors in the code.\newline
 Line 3 accounts for the math library.\newline
 Line 4 creates a list of all the files to compile with.\newline
 Line 5 makes .o files from the .c files.\newline
 Line 6 gives a name to the executable file.\newline
 Line 11 sets a template to compile the code with, similarly to how the code would be compiled without the Makefile.\newline
 Line 14 will compile the .c files to the respective .o files.\newline
 Line 17 will remove objects made via the compiler.\newline\newline
 To run the makefile use 'make' in the terminal, to use the clean function use 'make clean'.
 

    \newpage
	\section{TABLES}
    
 	\begin{table}[h!]
		\caption{Results with Crossover Rate = 0.5 and Mutation Rate = 0.05}
		\label{table:1}
		\centering
		\begin{tabular}{c c c c c c}
			\hline
			Pop Size & Max Gen & \multicolumn{3}{c}{Best Solution} & CPU time (Sec) \\
			& & $x_1$ & $x_2$ & Fitness & \\
			\hline
			10  & 100    &-0.022005&-0.2722371&1.945565&0.000409\\
			100 & 100    &-0.045144&0.038342&0.258794&0.000804\\
			1000& 100    &0.058413&-0.005530&0.255090&0.018999\\
			10000& 100    &-0.035636&-0.019404&0.158102&0.096210\\
			\hline
			1000  & 1000   &-0.007779&0.007002&0.032516&0.094375\\
			1000 & 10000  &0.001359&-0.003545&0.011121&0.877879\\
			1000& 100000 &0.001559&-0.000773&0.005004&7.780251\\
			1000& 1000000 &0.000489&0.000491&0.001973&83.814786\\
			\hline
		\end{tabular}
	\end{table}

  	\begin{table}[h!]
		\caption{Results with Crossover Rate = 0.5 and Mutation Rate = 0.2}
		\label{table:2}
		\centering
		\begin{tabular}{c c c c c c}
			\hline
			Pop Size & Max Gen & \multicolumn{3}{c}{Best Solution} & CPU time (Sec) \\
			& & $x_1$ & $x_2$ & Fitness & \\
			\hline
			10  & 100    &-0.209963&-0.013136&1.440489&0.000482\\
			100 & 100    &-0.021445&0.117224&0.675457&0.004225\\
			1000& 100    &-0.003721&0.032372&0.120194 &0.017529\\
			10000& 100    &0.007886&0.004132&0.027291&0.113018\\
			\hline
			1000  & 1000   &-0.006082&-0.000118&0.124004&0.124004\\
			1000 & 10000  &0.000284&0.002386&0.006951&0.907742\\
			1000& 100000 &0.000154&-0.000192&0.000697&9.123633\\
			1000& 1000000 &-0.000134&-0.000248&0.000800&100.425645\\
			\hline
		\end{tabular}
	\end{table}

    \newpage
	\section{HOW TO COMPILE}
 To compile the code ensure functions.c, OF.c, GA.c, functions.h, and Makefile are in the same directory.\newline\newline
 To compile the files, use the terminal and type 'make' this will compile the code. \newline\newline
 To run the code after compiling use the following parameters:\newline
 ./GA  population\_size  max\_generations  crossover\_rate  mutate\_rate  stop\_criteria\newline\newline
 An example of a proper input is: ./GA 1000 10000 0.5 0.1 1e-16\newline\newline\newline
 NOTE: While compiling you might receive an error regarding M\_PI and M\_E. This is because they are undefined. To fix this set M\_PI and M\_E to the values of pi and e respectively. 	
\end{document}